\chapter*{Abstract}\label{abstract}
This report aims to discuss existing literature and determine the future course of action for the progression of the thesis, alongside the consideration of project-dependent preparations to supplement the in-progression honours thesis, as a requirement of software engineering. 

Existing gaps within the literature consisted of gaps within the integration of smart home technologies with computer vision techniques for home automation.
There is a significant amount of research into human action recognition and behavioural monitoring in ambient assisted living environments for the care of the elderly and disabled but none covers the integration of this into smart home automation and predictive device control.

Project-dependent preparations were determined early on, selected in conjunction with the main thesis aims, these being the implementation of computer vision software for human action recognition, training an AI model to predict user movements and control devices accordingly, and developing a front-end web interface for manual device control and gesture registering.

Overall, the preparations aim to support the thesis aims and assist in the development of knowledge to achieve the outcomes whilst allowing for the research to be undertaken to the highest possible standard. 

Preliminary progress involved the trialing of the Robot Operating System (ROS) with the goal of integrating a Kinect depth sensor with ROS and controlling an external device through data from the Kinect, as a proof of concept for the project.

The planned progress was put forth to detail the plans for future progress of the work across thesis B and C. The current outline suggests week-by-week progress of the thesis aims outlined above, with each step completed within two weeks and evaluation of the system completed by week 8 of term 2. This is subject to change, determined by weekly progress.