\section{Problem Statement (Draft)}

Smart homes today promise enhanced convenience, safety, and security through automation and connectivity.
However, several critical challenges hinder the industry's growth and its overall effectiveness as a tool.

Interoperability issues between devices and existing platforms, the lack of genuine "intelligence" in these automated systems, and the constant reliance on internet access pose significant obstacles to realising the true potential of smart home technology.

In response to these challenges, this thesis aims to develop an entirely custom and customisable, intelligent environment where a user's movements can accurately predict how to control devices within the home.
Utilising local processing and sensor data, this system seeks to enhance convenience, safety, and security without relying on internet connectivity or support from third-party companies to allow interoperability.

The research objectives include designing and implementing artificial intelligence and computer vision for movement prediction and gesture recognition, integrating these into existing smart home infrastructure, and evaluating the performance and user experience of the environment.

The anticipated outcomes of this research include advancements in smart home technology, improved user experiences, and insights into addressing the broader challenges in the field as outlined in the literature review.
While the research focuses on addressing specific aspects of smart home functionality, certain limitations, such as device compatibility and privacy concerns, will be acknowledged and considered throughout the study.

Overall, this research endeavours to contribute to the evolution of smart homes by introducing innovative approaches to enhance their intelligence and autonomy, ultimately improving the quality of life for users.