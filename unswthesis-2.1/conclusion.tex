\chapter{Conclusion}\label{ch:conclusion}

This report has covered the evolution of smart homes since the 20th century, and how it has reached the current state of the art in smart home technology, as well as existing literature exploring the potential implementations of computer vision technology and human action recognition in future smart homes.
Behaviour monitoring in ambient assisted living environments has also been considered for use as an advanced technology with potential applications in home automation.
The gaps in the literature show that there is a vast amount of research into each of these fields but very little in actual application and deployment in controlling IoT devices in the home.

A detailed plan for Thesis~B predicts an eight-week timeline for the completion of the project, with a focus on the implementation of the CV gesture recognition system and front-end for manual device control.

The progress during Thesis~A so far, from the primitive integration of the Kinect depth sensor with ROS 2, has shown that the proposed system is viable and can be implemented over the course of the next term.

\section{Future Work}
The future work will be continued as outlined in Chapter~\ref{ch:plan}.
At the conclusion of this thesis, further research can be undertaken to explore multi-modal inputs for gesture and voice recognition in conjunction to allow for more natural interaction with the smart home.