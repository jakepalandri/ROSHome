\chapter*{Abstract}\label{abstract}

With the ever-increasing prevalence of smart devices in the home, the need for a more intuitive and efficient way to control these devices has become apparent.
This thesis aims to develop a system that uses computer vision techniques to recognise gestures, in conjunction with voice commands, to control a variety of smart appliances.

Existing literature in the space of smart home automation and computer vision has been reviewed to inform the design and development of the system.
Existing gaps within the literature consisted of gaps within the integration of smart home technologies with computer vision techniques for home automation.
There is a significant amount of research into human action recognition and behavioural monitoring in ambient assisted living environments for the care of the elderly and disabled but none of which covers the integration of this technology into smart home automation.

The system has been designed to be user-friendly and accessible to all members of the household, regardless of their technical expertise, which was another gap noted in existing deployments.
The system has been evaluated based on its accuracy, ease of use, and overall effectiveness in controlling smart devices.
The results of this evaluation were used to inform future research and development in the field of smart home automation.

The results indicated that the system was able to accurately recognise gestures with 97\% accuracy and voice commands with 68\% accuracy under normal conditions.
With background speaking, the voice command accuracy dropped to less than 10\%.
The system was able to control a variety of smart devices, including lights, and televisions, with the ability to expand to control other devices in the future without significant work from the users.