\chapter{Introduction}\label{ch:intro}

Smart home technology has seen an exponential rate of growth over the last century.
The technology has evolved from simple home appliances, to automatic control systems for a wide range of home devices.
In particular, each iteration of new smart home technology has evolved to be more user-friendly and more integrated with the user's daily life, freeing up time for more leisure activities, while automating menial tasks.

This report will cover existing literature and the investigation into the future of smart home technology, considering computer vision and human action recognition techniques for the purpose of home automation.
Additionally, the implementation of custom voice controlled smart home devices will be discussed, as well as the potential for multi-modal control of smart home devices.

Some of the key challenges that have been identified include the interoperability of smart home devices, the requirement for constant internet access, and the need for a more accessible and customisable interface for users without technical expertise.
The system that will be developed in this thesis aims to address these challenges by allowing users robust control over the voice commands and devices available to them, without the need to modify the code for the system with each new device that is added to the home.

\newpage

Each chapter of the report covers the topics as follows:
\begin{itemize}
    \item Chapter~\ref{ch:lit_review} explains the background for this thesis and related research.
    \item Chapter~\ref{ch:methodology} outlines the functionality of the system and how each component interacts with the others.
    \item Chapter~\ref{ch:evaluation} analyses the problems that the implemented system solves and discusses the results of user testing.
    \item Chapter~\ref{ch:conclusion_future_work} summarises the report.
    \item Chapter~\ref{ch:appendix} provides a user manual for the system so that it can be replicated by others.
\end{itemize}
